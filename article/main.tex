\documentclass[manuscript, screen, review]{jair}

\setcopyright{cc}
\copyrightyear{2026}
\acmDOI{10.1613/jair.1.xxxxx}

\JAIRAE{Insert JAIR AE Name}
\JAIRTrack{}
\acmVolume{}
\acmArticle{}
\acmMonth{}
\acmYear{2026}

\usepackage[
  datamodel=acmdatamodel,
  style=acmauthoryear,
  backend=biber,
  giveninits=true,
  uniquename=init
  ]{biblatex}

\renewcommand*{\bibopenbracket}{(}
\renewcommand*{\bibclosebracket}{)}

\addbibresource{references.bib}

%% Additional packages
\usepackage{booktabs}
\usepackage{pifont}
%% graphicx and xcolor already loaded by acmart.cls
\usepackage{listings}
\usepackage{subcaption}
\usepackage{multirow}
\usepackage{enumitem}

\lstset{
  basicstyle=\ttfamily\small,
  frame=single,
  breaklines=true,
  columns=fullflexible,
  keepspaces=true,
  numbers=left,
  numberstyle=\tiny\color{gray},
  numbersep=5pt,
  xleftmargin=15pt,
}

\begin{document}

\title[Hidden Non-Determinism in LLM APIs]{Hidden Non-Determinism in Large Language Model APIs: A Lightweight Provenance Protocol for Reproducible Generative AI Research}

\author{Lucas Rover}
\authornote{Corresponding Author.}
\orcid{0000-0001-6641-9224}
\email{lucasrover@utfpr.edu.br}
\affiliation{%
  \institution{UTFPR -- Universidade Tecnol\'ogica Federal do Paran\'a}
  \department{Programa de P\'os-Gradua\c{c}\~ao em Engenharia Mec\^anica}
  \city{Ponta Grossa}
  \state{Paran\'a}
  \country{Brazil}
}

\author{Yara de Souza Tadano}
\orcid{0000-0002-3975-3419}
\email{yaratadano@utfpr.edu.br}
\affiliation{%
  \institution{UTFPR -- Universidade Tecnol\'ogica Federal do Paran\'a}
  \department{Programa de P\'os-Gradua\c{c}\~ao em Engenharia Mec\^anica}
  \city{Ponta Grossa}
  \state{Paran\'a}
  \country{Brazil}
}

\renewcommand{\shortauthors}{Rover and Tadano}

\begin{abstract}
\textbf{Background:}
Generative AI models produce non-deterministic outputs that vary across runs, even under nominally identical configurations. This variability threatens the reproducibility of studies that rely on large language model (LLM) outputs, yet most existing experiment-tracking tools were not designed for the specific challenges of text-generation workflows.

\textbf{Objectives:}
We propose a lightweight, open-standard protocol for logging, versioning, and provenance tracking of generative AI experiments. The protocol introduces two novel documentation artifacts---Prompt Cards and Run Cards---and adopts the W3C PROV data model to create auditable, machine-readable provenance graphs linking every output to its full generation context.

\textbf{Methods:}
We formalize the protocol and evaluate it empirically through 1,864 controlled experiments. These experiments employ two models---LLaMA~3 8B (locally deployed) and GPT-4 (cloud API)---on two NLP tasks (scientific summarization and structured extraction) across 30 scientific abstracts and five experimental conditions that systematically vary the seed, temperature, and decoding strategy. We measure output variability using Exact Match Rate, Normalized Edit Distance, ROUGE-L, and BERTScore, and quantify the protocol's own overhead in terms of time and storage.

\textbf{Results:}
Under greedy decoding ($t{=}0$), LLaMA~3 achieves near-perfect reproducibility on extraction (EMR = 0.987) and summarization (EMR = 0.947). By contrast, GPT-4 under identical greedy settings achieves only EMR = 0.443 for extraction and EMR = 0.230 for summarization, revealing substantial server-side non-determinism that is invisible without systematic logging. Increasing temperature to 0.7 eliminates exact matches for both models. The protocol adds a mean overhead of 25.43\,ms per run (0.545\% of inference time) and approximately 4.1\,KB per run record, totaling 19.52\,MB for all 1,864 runs.

\textbf{Conclusions:}
Our results demonstrate that (1)~local inference is substantially more reproducible than API-based inference even under nominally identical parameters, (2)~structured output tasks are inherently more reproducible than open-ended generation, (3)~temperature is the dominant \textit{user-controllable} factor affecting variability, and (4)~comprehensive provenance logging can be achieved with negligible overhead. The protocol, reference implementation, and all experimental data are publicly available.
\end{abstract}

\begin{CCSXML}
<ccs2012>
<concept>
<concept_id>10011007.10011074.10011099.10011102.10011103</concept_id>
<concept_desc>Software and its engineering~Software testing and debugging</concept_desc>
<concept_significance>500</concept_significance>
</concept>
<concept>
<concept_id>10010147.10010257</concept_id>
<concept_desc>Computing methodologies~Machine learning</concept_desc>
<concept_significance>500</concept_significance>
</concept>
<concept>
<concept_id>10011007.10011074.10011099.10011692</concept_id>
<concept_desc>Software and its engineering~Documentation</concept_desc>
<concept_significance>300</concept_significance>
</concept>
</ccs2012>
\end{CCSXML}

\ccsdesc[500]{Software and its engineering~Software testing and debugging}
\ccsdesc[500]{Computing methodologies~Machine learning}
\ccsdesc[300]{Software and its engineering~Documentation}

\keywords{reproducibility, large language models, non-determinism, provenance, generative AI, experiment tracking, W3C PROV, scientific methodology}

\received{February 2026}

\maketitle

%% ============================================================
\section{Introduction}
\label{sec:introduction}

When a researcher queries GPT-4 with the same prompt and temperature zero, one would reasonably expect identical outputs. Our experiments show otherwise: across five controlled seeds under greedy decoding, GPT-4 produces the same extraction result only 44\% of the time. This hidden non-determinism exemplifies a fundamental challenge introduced by the rapid adoption of large language models (LLMs) in scientific research: how to ensure that studies relying on generative AI outputs are reproducible, auditable, and scientifically rigorous. Unlike traditional computational experiments, in which deterministic algorithms produce identical results given identical inputs, LLMs exhibit inherent variability in their outputs due to stochastic sampling, floating-point non-determinism, and opaque model-versioning practices \cite{chen2023chatgpt, zhu2023reproducibility}.

This reproducibility challenge is not merely theoretical. \citet{baker2016reproducibility} reported that over 70\% of researchers have failed to reproduce another scientist's experiment, a crisis that extends to AI research \cite{hutson2018artificial, gundersen2018state, stodden2016enhancing, kapoor2023leakage}. For generative AI specifically, the problem is compounded by several factors unique to text-generation workflows: (1)~the same prompt can yield semantically similar yet textually distinct outputs across runs; (2)~API-based models may undergo silent updates that alter behavior; (3)~temperature and sampling parameters create a high-dimensional space of possible outputs; and (4)~no established standard exists for documenting the full context needed to understand, audit, or reproduce a generative output.

Existing experiment-tracking tools such as MLflow \cite{zaharia2018accelerating}, Weights \& Biases \cite{biewald2020experiment}, and DVC \cite{kuprieiev2024dvc} were designed primarily for training pipelines and numerical metrics. Although valuable for their intended purposes, these tools lack features critical for generative AI studies: structured prompt versioning, cryptographic output hashing for tamper detection, provenance graphs linking outputs to their full generation context, and environment fingerprinting specific to inference-time conditions.

In this paper, we make three contributions:

\begin{enumerate}
    \item \textbf{A lightweight protocol} for logging, versioning, and provenance tracking of generative AI experiments. The protocol introduces \textit{Prompt Cards} and \textit{Run Cards} as structured documentation artifacts, and adopts the W3C PROV data model \cite{w3cprov2013} for machine-readable provenance graphs.

    \item \textbf{An empirical evaluation} of both the protocol's effectiveness and the reproducibility characteristics of LLM outputs. Through 1,864 controlled experiments with LLaMA~3 8B (local) and GPT-4 (API) across two tasks, 30 abstracts, and five conditions, we quantify output variability using three complementary metrics and measure the protocol's overhead. Our results reveal a striking reproducibility gap between local and API-based inference that is invisible without systematic logging.

    \item \textbf{A reference implementation} in Python that demonstrates the protocol's practical applicability, together with all experimental data, to facilitate adoption and independent verification.
\end{enumerate}

The remainder of this paper is organized as follows. Section~\ref{sec:related-work} reviews related work on reproducibility in AI and experiment tracking. Section~\ref{sec:protocol} formalizes the protocol design. Section~\ref{sec:experimental-setup} describes the experimental methodology. Section~\ref{sec:results} presents the empirical results. Section~\ref{sec:discussion} discusses findings, limitations, and practical implications. Section~\ref{sec:conclusion} concludes with directions for future work.


%% ============================================================
\section{Related Work}
\label{sec:related-work}

\subsection{Reproducibility in AI Research}

The reproducibility crisis in AI has been documented extensively. \citet{gundersen2018state} surveyed 400 AI papers and found that only 6\% provided sufficient information for full reproducibility. \citet{pineau2021improving} reported on the NeurIPS 2019 Reproducibility Program, which introduced reproducibility checklists and found significant gaps between reported and actual reproducibility. More recently, \citet{gundersen2024improving} described four institutional mechanisms adopted by JAIR---reproducibility checklists, structured abstracts, badges, and reproducibility reports---establishing a community standard for what should be documented in AI research. \citet{gundersen2018sources} identified three levels of reproducibility in AI---method, data, and experiment---and argued that all three are necessary for scientific progress. \citet{belz2021systematic} conducted a systematic review of 601 NLP papers and confirmed pervasive under-reporting of experimental details, while \citet{dodge2019show} proposed improved reporting standards for ML experiments, including confidence intervals and significance tests across multiple runs. More broadly, \citet{kapoor2023leakage} identified data leakage as a widespread driver of irreproducible results across 17 scientific fields that use ML-based methods.

For generative AI specifically, \citet{chen2023chatgpt} demonstrated that ChatGPT's outputs on NLP benchmarks exhibit non-trivial variability across identical queries, even with temperature set to zero. \citet{zhu2023reproducibility} showed that reproducibility degrades further when tasks involve subjective judgment, such as social computing annotations. Most recently, \citet{atil2024nondeterminism} systematically measured the non-determinism of five LLMs under supposedly deterministic settings across eight tasks, finding accuracy variations up to 15\% across runs and introducing the Total Agreement Rate (TAR) metric. \citet{ouyang2024nondeterminism} confirmed that temperature zero does not guarantee determinism in ChatGPT code generation. Most recently, \citet{yuan2025nondeterminism} traced such non-determinism to numerical precision issues in GPU kernels and proposed LayerCast as a mitigation strategy. Our work complements these studies in three specific ways. First, whereas prior studies (including Atil et al.'s five-model, eight-task study) measure variability post hoc, we provide a structured provenance protocol that enables \textit{prospective} documentation and audit---answering not only ``how much variability?'' but also ``why did these outputs differ?'' through cryptographic hashing and W3C PROV graphs. Second, we directly compare local and API-based inference on identical tasks with identical prompts, isolating the deployment paradigm as a variable---a comparison absent from prior work. Third, we quantify the overhead of systematic logging, demonstrating that the ``cost of knowing'' is negligible.

\subsection{Experiment Tracking Tools}

Several tools exist for tracking machine learning experiments, although none was designed specifically for generative AI text-output workflows:

\textbf{MLflow} \cite{zaharia2018accelerating} provides experiment tracking, model packaging, and deployment. It logs parameters, metrics, and artifacts, but focuses on training pipelines and numerical outcomes rather than text-generation provenance.

\textbf{Weights \& Biases} \cite{biewald2020experiment} offers experiment tracking with visualization dashboards. It supports prompt logging but lacks structured prompt versioning, cryptographic output hashing, and provenance graph generation.

\textbf{DVC} \cite{kuprieiev2024dvc} provides data versioning through git-like operations. While effective for dataset management, it does not address run-level provenance or prompt documentation.

\textbf{OpenAI Evals} \cite{openai2023evals} is a framework for evaluating LLM outputs against benchmarks. It provides structured evaluation but is tightly coupled to OpenAI's ecosystem and does not generate interoperable provenance records.

\textbf{LangSmith} \cite{langsmith2023} offers tracing and evaluation for LLM applications. It captures detailed execution traces but uses a proprietary format and requires cloud connectivity.

More broadly, \citet{bommasani2022opportunities} identified reproducibility as a key risk for foundation models, and \citet{liang2023holistic} proposed the HELM benchmark for holistic evaluation of language models, including robustness and fairness dimensions that complement our reproducibility focus. In the provenance space, \citet{padovani2025yprov} recently introduced yProv4ML, a framework that captures ML provenance in PROV-JSON format with minimal code modifications; our protocol shares the commitment to W3C PROV but targets the specific challenges of stochastic text generation rather than training pipelines.

Table~\ref{tab:comparison} provides a systematic feature-by-feature comparison of our protocol with these tools. The key distinction is not merely one of tooling but of \textit{scientific capability}: existing tools log what happened during training (parameters, metrics, artifacts), whereas our protocol enables answering questions that these tools cannot---specifically, whether two generative outputs are provably derived from identical configurations, which exact factor caused a divergence between non-identical outputs, and whether an output has been tampered with post-generation. These capabilities require the combination of cryptographic hashing, structured prompt documentation, and W3C PROV provenance graphs that no existing tool provides. In short, our contribution is not an alternative experiment tracker but a \textit{reproducibility assessment framework} designed for the unique challenges of stochastic text generation.

%% Table 4 - Comparison
\begin{table*}[t]
\centering
\caption{Comparison of our protocol with existing reproducibility tools and frameworks for GenAI experiments. Checkmarks (\ding{51}) indicate full support; tildes ($\sim$) indicate partial support; dashes (--) indicate no support.}
\label{tab:comparison}
\small
\begin{tabular}{@{}lcccccc@{}}
\toprule
\textbf{Feature} & \textbf{Ours} & \textbf{MLflow} & \textbf{W\&B} & \textbf{DVC} & \textbf{OpenAI Evals} & \textbf{LangSmith} \\
\midrule
Prompt versioning (Prompt Card) & \ding{51} & -- & $\sim$ & -- & $\sim$ & $\sim$ \\
Run-level provenance (W3C PROV) & \ding{51} & -- & -- & -- & -- & -- \\
Cryptographic output hashing & \ding{51} & -- & -- & \ding{51} & -- & -- \\
Seed \& param logging & \ding{51} & \ding{51} & \ding{51} & -- & \ding{51} & \ding{51} \\
Environment fingerprinting & \ding{51} & $\sim$ & $\sim$ & $\sim$ & -- & -- \\
Model weights hashing & \ding{51} & -- & $\sim$ & \ding{51} & -- & -- \\
Overhead $<$1\% of inference & \ding{51} & $\sim$ & $\sim$ & N/A & N/A & $\sim$ \\
Designed for GenAI text output & \ding{51} & -- & -- & -- & \ding{51} & \ding{51} \\
Open standard (PROV-JSON) & \ding{51} & -- & -- & -- & -- & -- \\
Local-first (no cloud dependency) & \ding{51} & \ding{51} & -- & \ding{51} & -- & -- \\
\bottomrule
\end{tabular}
\end{table*}


\subsection{Provenance in Scientific Computing}

Data provenance---the lineage of data through transformations---has a rich history in database systems and scientific workflows \cite{herschel2017survey}. The W3C PROV family of specifications \cite{w3cprov2013} provides a standardized data model for representing provenance as directed acyclic graphs of \textit{entities}, \textit{activities}, and \textit{agents}. \citet{samuel2022computational} applied provenance tracking to computational biology workflows, demonstrating its value for reproducibility. However, to our knowledge, no prior work has applied W3C PROV specifically to generative AI experiment workflows, in which the challenge involves not only tracking data lineage but also capturing the stochastic generation context that determines output variability.

Taken together, these gaps point to a clear need: a lightweight, standards-based protocol that bridges generative AI inference with the provenance infrastructure already established in scientific computing. The next section presents our design for such a protocol.


%% ============================================================
\section{Protocol Design}
\label{sec:protocol}

Our protocol addresses the question: \textit{What is the minimum set of metadata that must be captured for each generative AI run to enable auditing, reproducibility assessment, and provenance tracking?} We address this question through four complementary components.

\subsection{Scope and Design Principles}

The protocol is designed around three principles:

\begin{enumerate}
    \item \textbf{Completeness}: Every factor that can influence a generative output must be captured---prompt text, model identity and version, inference parameters, environment state, and timestamps.
    \item \textbf{Negligible overhead}: The logging process must not materially affect the experiment. We target $<$1\% overhead relative to inference time.
    \item \textbf{Interoperability}: All artifacts are stored in open, machine-readable formats (JSON, PROV-JSON), aligned with the FAIR (Findable, Accessible, Interoperable, Reusable) principles \cite{wilkinson2016fair}, to enable tool integration and long-term preservation.
\end{enumerate}

\subsection{Prompt Cards}
\label{sec:prompt-cards}

A \textit{Prompt Card} is a versioned documentation artifact that captures the design rationale and metadata for a prompt template used in experiments. Each Prompt Card contains:

\begin{itemize}
    \item \texttt{prompt\_id}: Unique identifier
    \item \texttt{prompt\_hash}: SHA-256 hash of the prompt text, enabling tamper detection
    \item \texttt{version}: Semantic version number
    \item \texttt{task\_category}: Classification of the task (e.g., summarization, extraction)
    \item \texttt{objective}: Natural-language description of what the prompt is designed to achieve
    \item \texttt{assumptions}: Explicit assumptions about inputs and expected behavior
    \item \texttt{limitations}: Known limitations or failure modes
    \item \texttt{target\_models}: Models for which the prompt was designed and tested
    \item \texttt{expected\_output\_format}: Description of the expected output structure
    \item \texttt{interaction\_regime}: Single-turn, multi-turn, or chain-of-thought
    \item \texttt{change\_log}: History of modifications
\end{itemize}

Prompt Cards serve two purposes: they document design intent (supporting human understanding) and they provide a citable, hashable reference for automated provenance tracking. The concept draws inspiration from Model Cards \cite{mitchell2019model}, Datasheets for Datasets \cite{gebru2021datasheets}, and model info sheets for reproducibility assessment \cite{kapoor2023leakage}, extending the structured-documentation paradigm to the prompt layer of the generative AI pipeline.

\subsection{Run Cards}
\label{sec:run-cards}

A \textit{Run Card} captures the complete execution context of a single generative AI run. Each Run Card records 22 core fields organized into five groups (the complete schema, including additional optional fields, is given in Appendix~\ref{app:schema}):

\begin{enumerate}
    \item \textbf{Identification}: \texttt{run\_id}, \texttt{task\_id}, \texttt{task\_category}, \texttt{prompt\_card\_ref}
    \item \textbf{Model context}: \texttt{model\_name}, \texttt{model\_version}, \texttt{weights\_hash}, \texttt{model\_source}
    \item \textbf{Parameters}: \texttt{inference\_params} (temperature, top\_p, top\_k, max\_tokens, seed, decoding\_strategy), \texttt{params\_hash}
    \item \textbf{Input/Output}: \texttt{input\_text}, \texttt{input\_hash}, \texttt{output\_text}, \texttt{output\_hash}, \texttt{output\_metrics}
    \item \textbf{Execution metadata}: \texttt{environment} (OS, architecture, Python version, hostname), \texttt{environment\_hash}, \texttt{code\_commit}, \texttt{timestamps}, \texttt{execution\_duration\_ms}, \texttt{logging\_overhead\_ms}, \texttt{storage\_kb}
\end{enumerate}

The separation of logging overhead from execution time is deliberate: it allows researchers to verify that the protocol itself does not confound experimental measurements.

\subsection{W3C PROV Integration}
\label{sec:prov}

Each experimental group (defined by a unique model--task--condition--abstract combination) is automatically translated into a W3C PROV-JSON document \cite{w3cprov2013} that expresses the generation provenance as a directed graph. The mapping defines:

\begin{itemize}
    \item \textbf{Entities}: Prompt, InputText, ModelVersion, InferenceParameters, Output, ExecutionMetadata
    \item \textbf{Activities}: RunGeneration (the inference execution)
    \item \textbf{Agents}: Researcher, SystemExecutor (the execution environment)
\end{itemize}

PROV relations capture the causal structure:
\begin{itemize}
    \item \texttt{used}: RunGeneration used Prompt, InputText, ModelVersion, InferenceParameters
    \item \texttt{wasGeneratedBy}: Output wasGeneratedBy RunGeneration
    \item \texttt{wasAssociatedWith}: RunGeneration wasAssociatedWith Researcher, SystemExecutor
    \item \texttt{wasAttributedTo}: Output wasAttributedTo Researcher
    \item \texttt{wasDerivedFrom}: Output wasDerivedFrom InputText
\end{itemize}

This standardized representation enables automated reasoning about experiment provenance, including detecting when two runs share identical configurations and identifying the specific factors that differ between non-identical outputs. An abbreviated example document is given in Appendix~\ref{app:prov-example}.

\subsection{Reproducibility Checklist}
\label{sec:checklist}

We provide a 15-item checklist organized into four categories---Prompt Documentation, Model and Environment, Execution and Output, and Provenance---that researchers can use to self-assess the reproducibility of their generative AI studies. The complete checklist is provided in Appendix~\ref{app:checklist}.

\subsection{Extensions for Advanced Workflows}
\label{sec:extensions}

While our empirical evaluation focuses on single-turn, single-model inference, the protocol's field schema is designed to accommodate more complex workflows through optional extension fields:

\begin{itemize}
    \item \textbf{Retrieval-Augmented Generation (RAG):} Additional fields for \texttt{retrieval\_query}, \texttt{retrieved\_documents} (with hashes), \texttt{retrieval\_model}, and \texttt{chunk\_strategy} enable tracing which external context influenced the output.
    \item \textbf{Tool use and function calling:} Fields for \texttt{tools\_available}, \texttt{tool\_calls} (with arguments and results), and \texttt{tool\_call\_hashes} capture the full tool-use chain.
    \item \textbf{Multi-turn dialogues:} A \texttt{conversation\_history\_hash} field and \texttt{turn\_index} enable linking each turn to the full conversation state.
    \item \textbf{Chain-of-thought / agent workflows:} A \texttt{parent\_run\_id} field supports hierarchical provenance graphs for multi-step reasoning chains.
\end{itemize}

These extensions are not evaluated in our current experiments but are specified in the reference implementation's schema to support future adoption in production LLM pipelines.

Having defined the protocol's components, we now evaluate it empirically along two dimensions: the reproducibility characteristics it reveals across different models and conditions, and the overhead it imposes on the experimental workflow.


%% ============================================================
\section{Experimental Setup}
\label{sec:experimental-setup}

We designed a controlled experiment to simultaneously evaluate (a)~the reproducibility characteristics of LLM outputs under varying conditions and (b)~the overhead imposed by our logging protocol.

\subsection{Models and Infrastructure}

We evaluate two models representing fundamentally different deployment paradigms:

\textbf{LLaMA~3 8B} \cite{grattafiori2024llama3}: A locally deployed open-weight model served through Ollama \cite{ollama2024} on an Apple M4 system with 24\,GB unified memory running macOS 14.6. Local deployment provides complete control over the execution environment, eliminating confounding factors such as network latency, server-side batching, and silent model updates. The software stack comprised Ollama v0.5.4, Python 3.12.8, the \texttt{ollama} Python SDK v0.4.7, and the LLaMA~3 8B Q4\_0 quantization (SHA-256 recorded per run).

\textbf{GPT-4} \cite{achiam2023gpt4}: A cloud-based proprietary model accessed via the OpenAI API (\texttt{openai} Python SDK v1.59.9) with controlled seed parameters. Although we requested \texttt{model="gpt-4"}, the API returned \texttt{gpt-4-0613} as the resolved model version in all runs, which we recorded in the \texttt{model\_id\_returned} field of each run record. This represents the typical deployment scenario where researchers have limited control over the inference environment. The API introduces additional sources of variability: load balancing, server-side batching, potential model-version updates, and floating-point non-determinism across different hardware.

\subsection{Tasks}

We evaluate two tasks that represent complementary points on the output-structure spectrum:

\textbf{Task 1: Scientific Summarization.} Given a scientific abstract, produce a concise summary in exactly three sentences covering the main contribution, methodology, and key quantitative result. This is an open-ended generation task in which the model has considerable freedom in word choice and phrasing.

\textbf{Task 2: Structured Extraction.} Given a scientific abstract, extract five fields (objective, method, key\_result, model\_or\_system, benchmark) into a JSON object. This is a constrained generation task in which the output format is fixed and the model must select, rather than generate, content.

\subsection{Input Data}

We use 30 widely-cited scientific abstracts from landmark AI/ML papers, including \citet{vaswani2017attention} (Transformer), \citet{devlin2019bert} (BERT), \citet{brown2020language} (GPT-3), \citet{raffel2020exploring} (T5), \citet{wei2022chain} (Chain-of-Thought), as well as seminal works on GANs, ResNets, VAEs, LSTMs, CLIP, DALL-E~2, Stable Diffusion, LLaMA, InstructGPT, PaLM, and others. These abstracts vary in length (74--227 words), technical complexity, and the number of quantitative results reported, thereby providing substantial diversity in the generation challenge.

\subsection{Experimental Conditions}

We define five conditions (Table~\ref{tab:experimental-design}) that systematically vary the factors hypothesized to affect reproducibility:

%% Table 1 - Experimental Design
\begin{table}[t]
\centering
\caption{Experimental design: conditions, parameters, and expected outcomes.}
\label{tab:experimental-design}
\small
\begin{tabular}{@{}llcccl@{}}
\toprule
\textbf{Cond.} & \textbf{Description} & \textbf{Temp.} & \textbf{Seed} & \textbf{Reps} & \textbf{Expected Outcome} \\
\midrule
C1 & Fixed seed, greedy & 0.0 & 42 (fixed) & 5 & Deterministic output \\
C2 & Variable seeds, greedy & 0.0 & 5 different & 5 & Near-deterministic \\
C3$_{t{=}0.0}$ & Temp.\ baseline & 0.0 & per-rep & 3 & Deterministic \\
C3$_{t{=}0.3}$ & Low temperature & 0.3 & per-rep & 3 & Low variability \\
C3$_{t{=}0.7}$ & High temperature & 0.7 & per-rep & 3 & High variability \\
\bottomrule
\end{tabular}

\par\smallskip\raggedright\footnotesize
\textit{Note:} Each condition is applied to 30~abstracts $\times$ 2~tasks = 60~groups per condition.
Total: 1,864 logged runs (1,140 LLaMA~3 + 724 GPT-4). GPT-4 C1 is severely incomplete (8/300 runs) due to API quota exhaustion; GPT-4 C1 extraction is omitted from Table~\ref{tab:variability-results}.
\end{table}

\textbf{C1 (Fixed seed, greedy decoding):} Temperature = 0, seed = 42 for all 5 repetitions. This represents the maximum-control condition and should yield deterministic outputs.

\textbf{C2 (Variable seeds, greedy decoding):} Temperature = 0, seeds = \{42, 123, 456, 789, 1024\}. This condition tests whether seed variation affects outputs when greedy decoding is used.

\textbf{C3 (Temperature sweep):} Three sub-conditions at $t \in \{0.0, 0.3, 0.7\}$ with 3 repetitions each, using different seeds per repetition. This condition characterizes how temperature affects output variability.

For LLaMA~3, each task $\times$ abstract combination is evaluated under conditions C1 (5 runs), C2 (5 runs), and C3 (9 runs = 3 temperatures $\times$ 3 reps), yielding 19 runs per pair, or $19 \times 30 \times 2 = 1{,}140$ runs. For GPT-4, all three conditions are included: C1 (5 runs), C2 (5 runs), and C3 (9 runs) per pair, or $19 \times 30 \times 2 = 1{,}140$ planned runs; due to API quota exhaustion, 724 valid runs were collected (C2: 300/300 complete; C3: 416/450; C1: 8/300). \textbf{Total: 1,864 valid runs.}

\subsection{Metrics}
\label{sec:metrics}

We adopt an operational definition of reproducibility at three levels, each mapped to a specific metric:

\begin{itemize}
    \item \textbf{Exact reproducibility} (string-level): Two outputs are identical character-by-character. Measured by \textit{Exact Match Rate (EMR)}.
    \item \textbf{Near reproducibility} (edit-level): Two outputs differ only in minor surface variations (punctuation, whitespace, synonym substitution). Measured by \textit{Normalized Edit Distance (NED)}.
    \item \textbf{Semantic reproducibility} (meaning-level): Two outputs convey the same information despite different phrasing. Measured by \textit{ROUGE-L F1} and, for a subset of conditions, \textit{BERTScore F1}.
\end{itemize}

This three-level framework allows us to distinguish between outputs that are bitwise identical (EMR~=~1), textually close (NED~$<$~0.05), and semantically equivalent (ROUGE-L~$>$~0.90). We measure output variability using these complementary metrics, computed over all pairwise comparisons within each condition group:

\textbf{Exact Match Rate (EMR):} The fraction of output pairs that are character-for-character identical. EMR = 1.0 indicates perfect reproducibility; EMR = 0.0 indicates that no two outputs match exactly.

\textbf{Normalized Edit Distance (NED):} The Levenshtein edit distance \cite{levenshtein1966binary} between each pair, normalized by the length of the longer string. NED = 0.0 indicates identical outputs; higher values indicate greater textual divergence.

\textbf{ROUGE-L F1:} The F1 score based on the longest common subsequence at the word level \cite{lin2004rouge}. This captures semantic similarity even when surface forms differ. ROUGE-L = 1.0 indicates identical word sequences.

Our primary metrics (EMR, NED, ROUGE-L) focus on exact and near reproducibility, which are the most direct measures for our research question. To complement these surface-level metrics, we also compute \textbf{BERTScore F1} \cite{zhang2020bertscore}---an embedding-based semantic similarity metric---for all conditions. BERTScore captures meaning-level equivalence that surface metrics may miss (e.g., paraphrases), providing a fourth perspective on reproducibility. For the structured extraction task, we additionally report \textbf{JSON validity rate}, \textbf{schema compliance rate}, and \textbf{field-level accuracy}, which measure whether outputs are syntactically valid JSON, contain all expected fields, and agree on individual field values across runs, respectively.

For protocol overhead, we measure:
\begin{itemize}
    \item \textbf{Logging time}: Wall-clock time spent on hashing, metadata collection, and file I/O, measured separately from inference time.
    \item \textbf{Storage}: Size of each run record (JSON) and total storage for all protocol artifacts.
    \item \textbf{Overhead ratio}: Logging time as a percentage of total execution time.
\end{itemize}


%% ============================================================
\section{Results}
\label{sec:results}

\subsection{Output Variability}

Table~\ref{tab:variability-results} presents the main variability results for both models, aggregated across all 30 abstracts.

%% Table 2 - Main Results (both models)
\begin{table*}[t]
\centering
\caption{Output variability across experimental conditions for LLaMA~3 8B (local) and GPT-4 (API). Mean over 30~abstracts. EMR = Exact Match Rate, NED = Normalized Edit Distance, ROUGE-L = word-level LCS F1, BS-F1 = BERTScore F1.}
\label{tab:variability-results}
\small
\begin{tabular}{@{}lllcccc@{}}
\toprule
\textbf{Model} & \textbf{Task} & \textbf{Condition} & \textbf{EMR}$\uparrow$ & \textbf{NED}$\downarrow$ & \textbf{ROUGE-L}$\uparrow$ & \textbf{BS-F1}$\uparrow$ \\
\midrule
  \multirow{10}{*}{\rotatebox[origin=c]{90}{LLaMA~3 8B}} & Summarization & C1 (fixed seed, $t{=}0$) & 0.947 & 0.0050 & 0.9945 & 0.9990 \\
   &  & C2 (same params, $t{=}0$) & 0.947 & 0.0050 & 0.9945 & 0.9990 \\
   &  & C3 ($t{=}0.0$) & 0.911 & 0.0083 & 0.9909 & 0.9984 \\
   &  & C3 ($t{=}0.3$) & 0.000 & 0.2790 & 0.7441 & 0.9669 \\
   &  & C3 ($t{=}0.7$) & 0.000 & 0.4438 & 0.5589 & 0.9432 \\
\cmidrule{2-7}
   & Extraction & C1 (fixed seed, $t{=}0$) & 0.987 & 0.0031 & 0.9966 & 0.9997 \\
   &  & C2 (same params, $t{=}0$) & 0.987 & 0.0031 & 0.9966 & 0.9997 \\
   &  & C3 ($t{=}0.0$) & 0.978 & 0.0052 & 0.9943 & 0.9996 \\
   &  & C3 ($t{=}0.3$) & 0.211 & 0.1224 & 0.8838 & 0.9851 \\
   &  & C3 ($t{=}0.7$) & 0.000 & 0.2530 & 0.7719 & 0.9693 \\
\midrule
  \multirow{10}{*}{\rotatebox[origin=c]{90}{GPT-4 (API)}} & Summarization & C1 (fixed seed, $t{=}0$) & 0.111 & 0.0879 & 0.8963 & 0.9897 \\
   &  & C2 (same params, $t{=}0$) & 0.230 & 0.1365 & 0.8695 & 0.9839 \\
   &  & C3 ($t{=}0.0$) & 0.144 & 0.1623 & 0.8479 & 0.9804 \\
   &  & C3 ($t{=}0.3$) & 0.000 & 0.2832 & 0.7238 & 0.9662 \\
   &  & C3 ($t{=}0.7$) & 0.000 & 0.4366 & 0.5554 & 0.9477 \\
\cmidrule{2-7}
   & Extraction & C2 (same params, $t{=}0$) & 0.443 & 0.0724 & 0.9384 & 0.9904 \\
   &  & C3 ($t{=}0.0$) & 0.381 & 0.0721 & 0.9356 & 0.9900 \\
   &  & C3 ($t{=}0.3$) & 0.143 & 0.1477 & 0.8669 & 0.9799 \\
   &  & C3 ($t{=}0.7$) & 0.000 & 0.2247 & 0.7890 & 0.9708 \\
\bottomrule
\end{tabular}
\end{table*}

\subsubsection{LLaMA~3 8B (Local Inference)}

\textbf{Finding 1: Structured extraction achieves near-perfect reproducibility under greedy decoding.} With $t = 0$, extraction produces EMR = 0.987 and NED = 0.0031 across conditions C1 and C2, meaning virtually every output is character-for-character identical. Summarization achieves an EMR of 0.947 with NED = 0.0050, indicating near-perfect but not complete reproducibility.

\textbf{Finding 2: Seed variation has no effect under greedy decoding.} Conditions C1 and C2 produce identical results despite using different seeds. With $t = 0$, the model always selects the highest-probability token, making the seed irrelevant. This finding confirms that greedy decoding provides reliably deterministic inference with locally deployed models.

\subsubsection{GPT-4 (API Inference)}

\textbf{Finding 3: API-based inference is substantially less reproducible than local inference, even under greedy decoding.} This is the most striking result of our study. Under greedy decoding ($t = 0$) with controlled seeds, GPT-4 achieves only EMR = 0.230 for summarization and EMR = 0.443 for extraction---compared to LLaMA's 0.947 and 0.987, respectively, under the same C2 condition.

Table~\ref{tab:model-comparison} highlights this reproducibility gap directly.

\begin{table}[t]
\centering
\caption{Reproducibility comparison: LLaMA~3 8B (local) vs.\ GPT-4 (API) under greedy decoding ($t{=}0$). GPT-4 shows markedly lower reproducibility due to server-side non-determinism ($p < 0.05$ for EMR; see text for paired $t$-tests).}
\label{tab:model-comparison}
\small
\begin{tabular}{@{}llcc@{}}
\toprule
\textbf{Task} & \textbf{Metric} & \textbf{LLaMA~3 8B} & \textbf{GPT-4} \\
\midrule
  Summarization & EMR & 0.840 & 0.200 \\
   & NED & 0.0148 & 0.0718 \\
   & ROUGE-L & 0.9823 & 0.9295 \\
\midrule
  Extraction & EMR & 1.000 & 0.520 \\
   & NED & 0.0000 & 0.0343 \\
   & ROUGE-L & 1.0000 & 0.9748 \\
\bottomrule
\end{tabular}
\end{table}

This gap is not due to parameter differences: both models use $t = 0$ with the same seed. The observed variability is consistent with server-side factors that are invisible to the researcher, including hardware-level floating-point non-determinism across different GPU types in the serving cluster, request-batching and scheduling effects, and potential silent model updates during the experimental window. While our experimental design controls for user-side parameters, we note that a definitive causal attribution would require additional controls (e.g., temporal replication across days, endpoint-level comparisons) that are infeasible with opaque API infrastructure. \textit{Without systematic logging, this non-determinism would be entirely invisible.}

\subsubsection{Temperature Effects Across Models}

\textbf{Finding 4: Temperature is the dominant \textit{user-controllable} factor affecting variability.} Figure~\ref{fig:temperature} shows the relationship between temperature and output variability for both models.

\begin{figure}[t]
    \centering
    \includegraphics[width=\linewidth]{figures/fig_temperature_both_models.pdf}
    \caption{Effect of temperature on output variability for both models. (a)~ROUGE-L F1 decreases monotonically with temperature. (b)~Exact Match Rate: LLaMA~3 starts from near-perfect reproducibility at $t = 0$, whereas GPT-4 starts from a lower baseline; however, both degrade at comparable rates with increasing temperature.}
    \Description{Two line plots comparing four configurations (LLaMA summarization, LLaMA extraction, GPT-4 summarization, GPT-4 extraction) across three temperature values.}
    \label{fig:temperature}
\end{figure}

For LLaMA~3, increasing temperature from 0 to 0.7 reduces ROUGE-L from 0.991 to 0.559 (summarization) and from 0.994 to 0.772 (extraction). For GPT-4, the same increase reduces ROUGE-L from 0.848 to 0.555 (summarization) and from 0.936 to 0.789 (extraction). The \textit{relative} rate of degradation is comparable, but GPT-4 starts from a lower baseline owing to its inherent server-side non-determinism.

Notably, BERTScore F1 remains above 0.94 across all conditions, even when EMR drops to zero at $t = 0.7$ (Table~\ref{tab:variability-results}). This indicates that while textual outputs diverge substantially at higher temperatures, their semantic content remains highly similar. The gap between surface-level metrics (EMR, NED) and semantic metrics (BERTScore) underscores that non-determinism in LLM outputs is primarily a \textit{phrasing} phenomenon rather than a \textit{meaning} phenomenon---a distinction with important practical implications for downstream applications that tolerate paraphrase variation.

\subsection{Cross-Model Comparison}

Figure~\ref{fig:model-comparison} provides a direct visual comparison of the two models under greedy decoding.

\begin{figure*}[t]
    \centering
    \includegraphics[width=\textwidth]{figures/fig_model_comparison.pdf}
    \caption{Reproducibility under greedy decoding ($t = 0$), averaged across conditions C1, C2, and C3 at $t{=}0$: LLaMA~3 8B (local) vs.\ GPT-4 (API). LLaMA~3 achieves near-perfect to perfect reproducibility, while GPT-4 shows measurable variability across all metrics, particularly for summarization. Condition-specific values are given in Table~\ref{tab:variability-results}.}
    \Description{Three grouped bar charts comparing LLaMA and GPT-4 on EMR, NED, and ROUGE-L for summarization and extraction tasks.}
    \label{fig:model-comparison}
\end{figure*}

Figure~\ref{fig:heatmap} presents a comprehensive heatmap of EMR across all model-task-condition combinations.

\begin{figure}[t]
    \centering
    \includegraphics[width=\linewidth]{figures/fig_heatmap_both_models.pdf}
    \caption{Heatmap of Exact Match Rate across all experimental conditions. The left columns (LLaMA~3) show high EMR under greedy decoding, while the right columns (GPT-4) show lower EMR even at $t = 0$. The vertical black line separates the two models.}
    \Description{A heatmap with 5 rows (conditions) and 4 columns (model-task combinations), using a viridis colormap where yellow indicates high EMR and dark purple indicates low EMR.}
    \label{fig:heatmap}
\end{figure}

To quantify the reproducibility gap between local and API-based inference, we performed paired $t$-tests on per-abstract EMR values under condition C2 (greedy decoding, $t = 0$) across all 30 abstracts. For summarization, the difference is highly significant: $t(29) = 17.250$, $p < 0.0001$, Cohen's $d = 3.149$ (LLaMA~3 mean EMR $= 0.947$, 95\% CI $[0.895, 0.998]$; GPT-4 mean EMR $= 0.230$, 95\% CI $[0.157, 0.303]$). For extraction, the gap is equally significant: $t(29) = 8.996$, $p < 0.0001$, Cohen's $d = 1.642$ (LLaMA~3 EMR $= 0.987$, 95\% CI $[0.959, 1.000]$; GPT-4 EMR $= 0.443$, 95\% CI $[0.316, 0.571]$). Both effect sizes are very large ($d > 1.6$), confirming that the reproducibility difference is not only statistically significant but practically meaningful. All $p$-values survive Bonferroni correction at the per-family threshold $\alpha = 0.05/6 \approx 0.008$, and all differences in NED and ROUGE-L are also significant ($p < 0.0001$). A post hoc power analysis confirms that $n = 30$ abstracts provides statistical power $> 0.999$ for all primary comparisons \cite{cohen1988statistical}.

\textbf{Robustness check.} Since EMR values are bounded on $[0, 1]$ and the paired differences are not normally distributed (Shapiro--Wilk: $W = 0.894$, $p = 0.006$ for summarization; $W = 0.885$, $p = 0.004$ for extraction), we additionally report non-parametric Wilcoxon signed-rank tests. All results remain highly significant: summarization $W = 0.0$, $p < 0.001$; extraction $W = 0.0$, $p < 0.001$. The convergence of parametric and non-parametric tests confirms that the reproducibility gap is robust to distributional assumptions.

\subsection{Protocol Overhead}

Table~\ref{tab:overhead} presents the protocol's overhead metrics across all 1,864 runs.


\begin{table}[t]
\centering
\caption{Protocol overhead: logging time and storage costs for 330~runs (190 LLaMA~3 + 140 GPT-4).}
\label{tab:overhead}
\small
\begin{tabular}{@{}lrl@{}}
\toprule
\textbf{Metric} & \textbf{Value} & \textbf{Unit} \\
\midrule
\multicolumn{3}{l}{\textit{Logging time overhead}} \\
\quad Mean per run & 33.56 $\pm$ 5.68 & ms \\
\quad Min / Max & 12.85 / 51.20 & ms \\
\quad Total (330 runs) & 11074 & ms \\
\quad Mean overhead ratio & 0.694\% & of inference time \\
\quad Max overhead ratio & 1.621\% & of inference time \\
\midrule
\multicolumn{3}{l}{\textit{Storage overhead}} \\
\quad Run logs (330 files) & 1382 & KB \\
\quad PROV documents (331 files) & 1736 & KB \\
\quad Run Cards (330 files) & 454 & KB \\
\quad Total output & 4.87 & MB \\
\bottomrule
\end{tabular}
\end{table}


The protocol adds a mean overhead of \textbf{25.43\,ms} per run, representing \textbf{0.545\%} of the mean inference time. This is well within our target of $<$1\%. The overhead is dominated by SHA-256 hashing and environment metadata collection; JSON serialization and file I/O contribute minimally.

Storage overhead is similarly modest: each run record occupies approximately 4.1\,KB, and the complete set of 1,864 run logs, provenance documents, and Run Cards totals 19.52\,MB. Note that provenance documents are generated per experimental group (i.e., per unique model--task--condition--abstract combination), yielding 331 PROV-JSON files that aggregate the individual runs within each group.

Figure~\ref{fig:overhead} shows the overhead distribution broken down by model.

\begin{figure}[t]
    \centering
    \includegraphics[width=\linewidth]{figures/fig_overhead_both_models.pdf}
    \caption{Distribution of protocol overhead by model. Left: Absolute logging time (ms). Right: Overhead as a percentage of inference time. Overhead is comparable between local (LLaMA~3) and API (GPT-4) inference, consistently below 1.7\%.}
    \Description{Two box plots comparing overhead between LLaMA 3 and GPT-4. Both models show similar overhead distributions.}
    \label{fig:overhead}
\end{figure}


%% ============================================================
\section{Discussion}
\label{sec:discussion}

The preceding results paint a nuanced picture: local inference under greedy decoding is near-perfectly reproducible, but API-based inference exhibits substantial hidden variability that researchers cannot control. Temperature is the dominant user-controllable factor, and structured tasks are inherently more reproducible than open-ended ones. We now consider what these findings mean for research practice, what the protocol enables that was previously invisible, and where the current study's limitations lie.

\subsection{Implications for Reproducibility Practice}

Our results yield several actionable recommendations for researchers conducting generative AI experiments:

\textbf{Use greedy decoding with local models for maximum reproducibility.} Under $t = 0$ with LLaMA~3 (local), extraction achieved 98.7\% EMR and summarization reached 94.7\% EMR across 30 abstracts. This configuration should be the default for any study in which output consistency is critical.

\textbf{Be aware of API non-determinism.} Our most consequential finding is that GPT-4, even with $t = 0$ and a fixed seed, produces substantially variable outputs (EMR = 0.230 for summarization, 0.443 for extraction). Researchers using API-based models should \textit{never assume reproducibility} without verification, and should report multiple runs with variability metrics.

\textbf{Prefer structured output formats when possible.} The extraction task's consistently higher reproducibility across both models demonstrates that output-format constraints directly improve reproducibility. Researchers should consider whether their tasks can be reformulated as structured extraction rather than open-ended generation.

\textbf{Include warm-up runs for local models.} The per-abstract analysis revealed that the first inference call after model loading may differ from subsequent calls owing to cache initialization effects. Discarding the first run is a straightforward practice that improves measured reproducibility. Consequently, the LLaMA~3 summarization EMR of 0.947 may represent a conservative lower bound: with a warm-up run excluded, the effective EMR would approach 1.000 for the remaining repetitions. Future studies should incorporate an explicit warm-up run as part of their experimental protocol.

\textbf{Log comprehensively; the cost is negligible.} At 0.545\% overhead and approximately 4\,KB per run, there is no practical reason not to apply comprehensive logging. The cost of not logging---namely, the inability to detect the kind of API non-determinism documented herein---far exceeds the protocol's minimal requirements.

\subsection{Local vs.\ API Inference: A Reproducibility Gap}

The most significant finding of this study is the reproducibility gap between local and API-based inference. Under nominally identical greedy decoding conditions, LLaMA~3 (local) achieves EMR = 0.987 for extraction while GPT-4 (API) achieves only 0.443. For summarization, the gap is 0.947 vs.\ 0.230.

This gap has profound implications for the scientific use of API-based LLMs. \textit{Without systematic logging, a researcher using GPT-4 would have no way of knowing that their ``deterministic'' experiment produces different outputs across runs.} Since our experimental design controls all user-side parameters (temperature, seed, prompt, input), the observed variability strongly suggests opaque server-side factors as the primary source. Our protocol makes this hidden non-determinism visible, measurable, and documentable.

\subsection{Task-Dependent Reproducibility}

The difference between summarization and extraction reproducibility under identical conditions---observed consistently across both models---is, to our knowledge, the first empirical quantification of how task structure affects LLM output reproducibility. This finding suggests a spectrum ranging from highly constrained tasks (structured extraction, classification) to open-ended tasks (summarization, dialogue), with the degree of output-space constraint serving as a primary determinant. Notably, even GPT-4's extraction task (EMR = 0.443) substantially outperforms its summarization task (EMR = 0.230), confirming that this effect is not specific to any single model.

\subsection{The Role of Provenance}

The W3C PROV graphs generated by our protocol serve multiple purposes beyond simple audit trails:

\begin{enumerate}
    \item \textbf{Automated comparison}: By comparing PROV graphs of two runs, one can automatically identify which factors differed (e.g., same prompt and model but different temperatures), enabling systematic diagnosis of non-reproducibility.
    \item \textbf{Lineage tracking}: When outputs are used as inputs to downstream processes (e.g., summarization outputs fed into a meta-analysis), the provenance chain can be extended to trace any final result back to its full generation context.
    \item \textbf{Compliance}: For regulated domains (healthcare, legal, finance), PROV documents provide the formal evidence trail required by audit standards \cite{nist2023ai} and emerging regulations such as the EU AI Act \cite{euaiact2024}.
\end{enumerate}

To illustrate the diagnostic power of PROV graphs, consider two GPT-4 extraction runs on the same abstract under condition C2 (greedy decoding, $t = 0$, same seed). Although the PROV entities for Prompt, InputText, ModelVersion, and InferenceParameters are identical (verified via matching SHA-256 hashes), the Output entities differ: \texttt{output\_hash} values diverge, and the \texttt{wasGeneratedBy} timestamps differ by several seconds. The PROV graph thus automatically pinpoints the source of non-reproducibility: the only varying factor is the RunGeneration activity itself, confirming that the non-determinism originates server-side. This kind of automated differential diagnosis is infeasible without structured provenance records.

\subsection{Limitations}
\label{sec:limitations}

We organize threats to validity following standard categories:

\subsubsection{Internal Validity}

\textbf{Sample size and statistical power.} With $n = 30$ abstracts per condition, our study has adequate statistical power for the primary comparisons. A post hoc power analysis using the observed effect sizes ($d > 1.6$) and $\alpha = 0.05$ yields power $> 0.999$ for all primary comparisons \cite{cohen1988statistical}. However, for one secondary comparison (extraction EMR under C3$_{t=0.3}$, $d = 0.207$), power is low (0.084), meaning that subtler effects may go undetected in some conditions.

\textbf{Warm-up confound.} As noted in Section~\ref{sec:discussion}, the first LLaMA~3 inference after model loading may differ from subsequent calls due to cache initialization. This affects a small number of abstracts (4 of 30 for summarization), reducing the aggregate EMR from $\sim$1.0 to 0.947. It represents an uncontrolled confound in our experimental design.

\textbf{Prompt format confound.} LLaMA~3 was queried via Ollama's \texttt{/api/generate} endpoint (raw completion), whereas GPT-4 was queried via the OpenAI Chat Completions API (structured messages with system/user roles). This difference in prompt format is inherent to the deployment paradigms under study and mirrors real-world usage, but it means the observed reproducibility gap may partially reflect prompt-format effects rather than purely server-side factors.

\subsubsection{External Validity}

\textbf{Two models.} Our evaluation covers LLaMA~3 8B (local) and GPT-4 (API), representing two deployment paradigms but only one model per category. Other models---including Claude \cite{anthropic2024claude}, Gemini \cite{geminiteam2024gemini}, Mixtral, and larger or smaller LLaMA variants---may exhibit different reproducibility characteristics. Our findings about the local-vs-API gap should therefore be interpreted as a case study of this paradigm difference rather than a universal claim. The protocol itself is model-agnostic, and we note that \texttt{gpt-4-0613} is now a legacy snapshot; the very fact that newer model versions may behave differently illustrates exactly the kind of silent evolution that our protocol is designed to detect and document.

\textbf{Two tasks.} Summarization and extraction represent distinct points on the output-structure spectrum but do not cover the full range of generative AI applications (e.g., dialogue, code generation, reasoning chains). A broader task suite would strengthen generalizability.

\textbf{English-only, single domain.} Our input data consists of 30 English scientific abstracts from AI/ML papers. While this is a substantial and diverse sample within one domain, reproducibility characteristics may differ for other languages, domains (e.g., biomedical, social science), or document types.

\textbf{No multi-turn evaluation.} All experiments use single-turn interactions. Multi-turn dialogues introduce additional variability through conversation history, which our protocol logs but our experiments do not evaluate.

\subsubsection{Construct Validity}

\textbf{Surface-level metrics.} Our metrics (EMR, NED, ROUGE-L) capture textual rather than semantic similarity. Two outputs that are semantically equivalent but syntactically different will register as non-matching under EMR and partially divergent under NED. This is by design---our focus is on \textit{exact} reproducibility---but it means our results may overstate the practical impact of non-determinism for downstream applications where semantic equivalence suffices.

\subsubsection{Other Considerations}

\textbf{Privacy.} The protocol's environment metadata includes the machine hostname, which may reveal institutional information. Deployments in privacy-sensitive settings should anonymize this field.

\textbf{Computational cost.} The total cost was modest: $\sim$2 GPU-hours on a consumer laptop (Apple M4, 24\,GB) for 1,140 LLaMA~3 runs, plus 724 API calls to GPT-4. The carbon footprint is negligible at this scale, and the logging overhead (25\,ms per run) would not materially increase energy consumption even at thousands of runs.

\subsection{Protocol Minimality: An Ablation Analysis}
\label{sec:ablation}

To substantiate our claim that the protocol captures a \textit{minimal} set of metadata, we conducted an ablation analysis in which we systematically removed each field group from the protocol schema and assessed which audit questions became unanswerable. We defined 10 audit questions that a reproducibility-oriented researcher might ask (e.g., ``Can we verify the exact prompt used?'', ``Can we detect output tampering?'', ``Can we trace full provenance?'') and mapped each to the protocol fields required to answer it. For this analysis, we decomposed the Run Card's five sections into eight finer-grained field groups by separating cross-cutting concerns: Identification, Model Context, Parameters, Input Content, Output Content, Hashing (all SHA-256 digests), Environment, and Overhead (timing and storage metadata).

The results show that removing \textit{any} of these eight field groups renders at least one audit question unanswerable, confirming that no group is redundant. The Hashing group (SHA-256 hashes for prompts, inputs, outputs, parameters, and environment) has the highest information density: its removal affects 6 of 10 questions despite contributing only 410\,bytes per run. Conversely, the Overhead group (logging time metadata) is the least connected but remains necessary for overhead assessment. The complete ablation results are available in the project repository.

This analysis demonstrates that the protocol is \textit{minimal} in the sense that every field group is necessary for at least one audit capability, while the total overhead remains at approximately 4,052 bytes per run.

\subsection{Practical Costs and Adoption}

One concern with any new protocol is whether the adoption burden is justified. We address this concretely:

\begin{itemize}
    \item \textbf{Implementation effort}: Our reference implementation adds approximately 600 lines of Python (the protocol core) to an existing workflow. Integration requires 3--5 function calls per run.
    \item \textbf{Runtime cost}: 25\,ms per run, negligible compared to inference times of seconds to minutes for typical LLM calls.
    \item \textbf{Storage cost}: 4\,KB per run. Even at scale (10,000 runs), total storage is approximately 40\,MB---less than a single model checkpoint.
    \item \textbf{Learning curve}: The protocol uses standard JSON and W3C PROV, requiring no specialized knowledge beyond basic Python.
\end{itemize}

Against these modest costs, the protocol provides complete audit trails, automated provenance graphs, tamper-detectable outputs via cryptographic hashing, and structured metadata that enable systematic reproducibility analysis.


%% ============================================================
\section{Conclusion}
\label{sec:conclusion}

We presented a lightweight protocol for logging, versioning, and provenance tracking of generative AI experiments, introducing Prompt Cards and Run Cards as novel documentation artifacts and adopting the W3C PROV data model for machine-readable provenance graphs. Through 1,864 controlled experiments with LLaMA~3 8B (local) and GPT-4 (API) across 30 scientific abstracts and two NLP tasks, we demonstrated four key findings:

\begin{enumerate}
    \item \textbf{Local inference is substantially more reproducible than API-based inference.} Under identical greedy decoding settings, LLaMA~3 achieves EMR = 0.987 for extraction while GPT-4 achieves only 0.443, revealing substantial server-side non-determinism that is invisible without systematic logging (paired $t$-test: $p < 0.0001$, Cohen's $d > 1.6$).

    \item \textbf{Task structure is a primary determinant of reproducibility.} Structured extraction consistently outperforms open-ended summarization across both models, with the JSON format constraint reducing the model's output space.

    \item \textbf{Temperature is the dominant user-controllable factor.} Increasing from $t = 0$ to $t = 0.7$ reduces ROUGE-L from 0.991 to 0.559 (LLaMA summarization) and from 0.936 to 0.789 (GPT-4 extraction), while seed variation has no measurable effect under greedy decoding for local models.

    \item \textbf{Comprehensive provenance logging adds negligible overhead}: 0.545\% of inference time and approximately 4\,KB per run, thereby removing any practical argument against systematic documentation.
\end{enumerate}

These findings carry a broader implication: a significant portion of published research that relies on API-based LLMs may contain non-reproducible results without the authors' knowledge. The cost of systematic provenance logging---half a percent of inference time and four kilobytes per run---is trivially small compared to the cost of publishing non-reproducible science.

Looking ahead, we plan to (i)~expand the model suite to include Claude \cite{anthropic2024claude}, Gemini \cite{geminiteam2024gemini}, and open-weight models of varying sizes; (ii)~extend the task coverage to dialogue, code generation, and multi-turn interactions; and (iii)~develop automated reproducibility scoring based on provenance graph analysis. Ultimately, we envision a future in which every generative AI output carries a provenance certificate, and reproducibility metrics are reported alongside accuracy as a standard component of empirical evaluation.

The reference implementation, all 1,864 run records, provenance documents, and analysis scripts are publicly available to support adoption and independent verification.


%% ============================================================
\begin{acks}
This work was supported by UTFPR -- Universidade Tecnol\'ogica Federal do Paran\'a. The experiments were conducted using locally deployed open-weight models to ensure full reproducibility of the computational environment.
\end{acks}

%% ============================================================
%% Declarations
%% ============================================================

\section*{Data Availability Statement}
The reference implementation, all 1,864 run records (JSON), PROV-JSON provenance documents, Run Cards, Prompt Cards, input data, analysis scripts, and generated figures are publicly available at:
\begin{center}
\url{https://github.com/Roverlucas/genai-reproducibility-protocol}
\end{center}
The repository includes instructions for reproducing all experiments and regenerating all tables and figures from the raw data.

\section*{Author Contributions}
Following the CRediT (Contributor Roles Taxonomy) framework:
\textbf{Lucas Rover}: Conceptualization, Methodology, Software, Validation, Formal Analysis, Investigation, Data Curation, Writing -- Original Draft, Writing -- Review \& Editing, Visualization, Project Administration.
\textbf{Yara de Souza Tadano}: Supervision, Conceptualization, Methodology, Writing -- Review \& Editing, Project Administration.

\section*{Conflict of Interest}
The authors declare no conflicts of interest. This research was conducted independently at UTFPR with no external funding from commercial AI providers. The use of OpenAI's GPT-4 API was for research evaluation purposes only and does not constitute an endorsement.

\section*{Use of AI-Assisted Tools}
The authors used AI-assisted tools (Claude, Anthropic) during the preparation of this manuscript for language editing, code development support, and data analysis scripting. All AI-generated content was critically reviewed, validated, and revised by the authors, who take full responsibility for the accuracy and integrity of the final manuscript. The scientific design, experimental execution, interpretation of results, and intellectual contributions are entirely the authors' own work.

\printbibliography


%% ============================================================
\appendix

\section{Reproducibility Checklist}
\label{app:checklist}

The following checklist is designed for self-assessment of reproducibility in generative AI studies. Each item maps to a specific field or artifact in our protocol.

\subsection*{Prompt Documentation}
\begin{enumerate}
    \item Is the exact prompt text recorded and versioned? \hfill [Prompt Card: \texttt{prompt\_text}, \texttt{prompt\_hash}]
    \item Are design assumptions and limitations documented? \hfill [Prompt Card: \texttt{assumptions}, \texttt{limitations}]
    \item Is the expected output format specified? \hfill [Prompt Card: \texttt{expected\_output\_format}]
    \item Is the interaction regime documented (single/multi-turn)? \hfill [Prompt Card: \texttt{interaction\_regime}]
\end{enumerate}

\subsection*{Model and Environment}
\begin{enumerate}[resume]
    \item Is the model name and version recorded? \hfill [Run Card: \texttt{model\_name}, \texttt{model\_version}]
    \item Are model weights hashed for identity verification? \hfill [Run Card: \texttt{weights\_hash}]
    \item Is the execution environment fingerprinted? \hfill [Run Card: \texttt{environment}, \texttt{environment\_hash}]
    \item Is the source code version recorded? \hfill [Run Card: \texttt{code\_commit}]
\end{enumerate}

\subsection*{Execution and Output}
\begin{enumerate}[resume]
    \item Are all inference parameters logged? \hfill [Run Card: \texttt{inference\_params}]
    \item Is the random seed recorded? \hfill [Run Card: \texttt{inference\_params.seed}]
    \item Is the output cryptographically hashed? \hfill [Run Card: \texttt{output\_hash}]
    \item Are execution timestamps recorded? \hfill [Run Card: \texttt{timestamp\_start}, \texttt{timestamp\_end}]
    \item Is logging overhead measured separately? \hfill [Run Card: \texttt{logging\_overhead\_ms}]
\end{enumerate}

\subsection*{Provenance}
\begin{enumerate}[resume]
    \item Is a provenance graph generated per group? \hfill [PROV-JSON document]
    \item Are provenance documents in an interoperable format? \hfill [W3C PROV standard]
\end{enumerate}


\section{Run Card Schema}
\label{app:schema}

The complete Run Card schema, with data types and descriptions:

\begin{lstlisting}[caption={Run Card JSON schema (simplified).},label={lst:schema}]
{
  "run_id": "string (unique identifier)",
  "task_id": "string (task identifier)",
  "task_category": "string (e.g., summarization)",
  "prompt_hash": "string (SHA-256 of prompt)",
  "prompt_text": "string (full prompt text)",
  "input_text": "string (input to the model)",
  "input_hash": "string (SHA-256 of input)",
  "model_name": "string (e.g., llama3:8b)",
  "model_version": "string (e.g., 8.0B)",
  "weights_hash": "string (SHA-256 of weights)",
  "model_source": "string (e.g., ollama-local)",
  "inference_params": {
    "temperature": "float",
    "top_p": "float",
    "top_k": "integer",
    "max_tokens": "integer",
    "seed": "integer|null",
    "decoding_strategy": "string"
  },
  "params_hash": "string (SHA-256 of params)",
  "environment": {
    "os": "string",
    "os_version": "string",
    "architecture": "string",
    "python_version": "string",
    "hostname": "string",
    "timestamp": "ISO 8601 datetime"
  },
  "environment_hash": "string (SHA-256)",
  "code_commit": "string (git commit hash)",
  "researcher_id": "string",
  "affiliation": "string",
  "timestamp_start": "ISO 8601 datetime",
  "timestamp_end": "ISO 8601 datetime",
  "output_text": "string (model output)",
  "output_hash": "string (SHA-256 of output)",
  "output_metrics": "object (task-specific)",
  "execution_duration_ms": "float",
  "logging_overhead_ms": "float",
  "storage_kb": "float",
  "system_logs": "string (raw system info)",
  "errors": "array of strings"
}
\end{lstlisting}


\section{Example PROV-JSON Document}
\label{app:prov-example}

An abbreviated example of a PROV-JSON document generated for a single summarization run:

\begin{lstlisting}[caption={Abbreviated PROV-JSON for a summarization run.},label={lst:prov}]
{
  "prefix": {
    "genai": "https://genai-prov.org/ns#",
    "prov": "http://www.w3.org/ns/prov#"
  },
  "entity": {
    "genai:prompt_c9644358": {
      "prov:type": "genai:Prompt",
      "genai:hash": "c9644358805b...",
      "genai:task_category": "summarization"
    },
    "genai:model_llama3_8b": {
      "prov:type": "genai:ModelVersion",
      "genai:name": "llama3:8b",
      "genai:source": "ollama-local"
    },
    "genai:output_590d0835": {
      "prov:type": "genai:Output",
      "genai:hash": "590d08359e7d..."
    }
  },
  "activity": {
    "genai:run_llama3_8b_sum_001_C1_rep0": {
      "prov:type": "genai:RunGeneration",
      "prov:startTime": "2026-02-07T21:54:34Z",
      "prov:endTime": "2026-02-07T21:54:40Z"
    }
  },
  "wasGeneratedBy": {
    "_:wGB1": {
      "prov:entity": "genai:output_590d0835",
      "prov:activity": "genai:run_llama3_8b_..."
    }
  },
  "used": {
    "_:u1": {
      "prov:activity": "genai:run_llama3_...",
      "prov:entity": "genai:prompt_c9644358"
    }
  },
  "agent": {
    "genai:researcher_lucas_rover": {
      "prov:type": "prov:Person",
      "genai:affiliation": "UTFPR"
    }
  },
  "wasAssociatedWith": {
    "_:wAW1": {
      "prov:activity": "genai:run_llama3_...",
      "prov:agent": "genai:researcher_..."
    }
  }
}
\end{lstlisting}


\end{document}
